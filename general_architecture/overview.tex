\section{General overview}
We will here be giving a general overview of the project. 

The project is mainly divided in three different parts, and the general 
communication pattern is shown in figure~\ref{fig:project-overview}.
\begin{figure}[tb]
  \begin{center}
    \begin{tikzpicture}[
  system/.style={draw,rectangle,rounded corners=3,minimum width=2cm,text width=1.8cm,text centered},
  node distance=2cm
  ]
  \draw (-7, -2.5) rectangle + (10,3.5);
  \draw (-6, -3) node  {Remote computer};
  
  \draw (-1.5, -6.3) rectangle + (3,1.5);
  \draw (2.8, -4.5) node {Local machine};

  \node[system] (sf) {Server front app};
  \node[system, below=4.8cm of sf] (fe) {Front end};
  \node[system, left=3.5cm of sf] (sb) {Server back app};
  \node[system, below=1cm of sb] (xc) {X client};

  \draw[style={->,>=triangle 45}] 
  ($(sb.east)+(0,-0.3)$) -- 
  ($(sf.west)+(0,-0.3)$)
  node [midway, below] {Transfer requests};
  
  \draw[style={->,>=triangle 45}] 
  ($(sf.west)+(0,0.3)$) --
  ($(sb.east)+(0,0.3)$)
  node [midway, above] {Transfer data};

  \draw[style={->,>=triangle 45}]
  ($(sb.south)+(-0.4,0)$) -- 
  ($(xc.north)+(-0.4,0)$);
  
  \draw[style={->,>=triangle 45}] 
  ($(xc.north)+(0.4,0)$) -- 
  ($(sb.south)+(0.4,0)$);

  \draw[style={->,>=triangle 45}]
  ($(fe.north)+(-0.5,0)$)--
  ($(sf.south)+(-0.5,0)$)
  node [midway, left, align=right] {Replies\\Events};
  
  \draw[style={->,>=triangle 45}] 
  ($(sf.south)+(0.5,0)$) --
  ($(fe.north)+(0.5,0)$)
  node [midway, right, align=left] {Requests};
\end{tikzpicture}
    \caption{\label{fig:project-overview}System general communication pattern}
  \end{center}
\end{figure}
\begin{description}
\item[Server backend]
  This application main role is to communicate with the X clients, 
  to transfer the requests sent by the X clients to the frontend of the system, 
  and to transfer back the replies and events sent by the frontend to the 
  X clients. 
%  
  % This sentence is little complex. You can explain more simply.
  Another important role of this application is to store as possible 
  the information held by the client and the frontend in order to, whenever 
  possible avoid unneeded communication by responding directly without 
  needing to transfer any information.
%
  % Below sentences also may be able to be more simply.
  % Especially, second sentence includes multiple sentences using ','.
  To illustrate this caching system, let us think about a requests sent by a 
  client to get the background and foreground pixels of a particular 
  subwindow (the root window information being sent on connection). If the 
  backend server was doing any cache work, the request would have to be 
  sent to the frontend application, and then to the client web browser, before 
  getting an answer, which would result in a, depending on the bandwith, 
  quite long latency for a very basic request.   
  
\item[Server frontend]
  One of the roles of this application  is to act as a bridge to
  communicate between the backend application of the system 
  and the web browser JavaScript application.
%
  Another of its role is to act as a login manager. The application 
  receives an HTTP request with the client Linux/UNIX creditentials 
  check those creditentials, and on success, initializes the server 
  backend for the particular client as well as the websocket connection 
  to communicate with the web browser.
\item[Web browser application]
  The web browser application is the application running in the client 
  browser, and has several responsibilities in the system.
%
  The first responsibility of this application is to handle the 
  requests received from the server and to render them in the web 
  browser.
%
  The second responsibility is to handle events occuring in the browser, and to 
  send them back to the server, for them to be handled by the X client.
\end{description}



