\section{Language and tool choice}
%
% I cannot understand the meaning of below sentences.
% You do not use DOM libraries.
% However (?) you use KineticJS.
% I think you should explain why you use KineticJS and what are features of KineticJS.
The web browser application is written in JavaScript, and we made the 
choice to avoid any DOM libraries that could potentially slow down the 
application and would not be very useful in this context as 
the application is almost only using Canvas.

For rendering Canvas, the application uses KineticJS\footnote{\url{http://www.kineticjs.com/}}. 
KineticJS is a Canvas library that allows layering and easy event handling.
%

The application running on the remote server has been almost entirely developed in Scala. 
In this section, we will try to discuss objectively about 
the motivations for choosing this language. Some of the arguments given 
are based not only on Scala but also on the library Akka which integrates 
very well with the design and syntax of Scala.

In the following paragraphs, we will only give some quite abstract arguments, 
to avoid entering in too many details. % A more detailed presentation 
% of the actor model and its integration in Scala can be found 
% in the appendix~\ref{chap:actor-model}.
%
\subsection{Multi-threading}
The system is developed to of course support several X client connections 
for a single user, but also to accept several user at the same time.
This therefore requires a fairly high amount of threads.

Multithreading is often the source of numerous mistakes when used naively 
using libraries such as pthread for C and C++, or \lstinline{java.lang.Thread} 
for Java. One of the main reason for this is that it is quite difficult to 
efficiently keep trace of the threads accessing a given resource, 
and this can sometimes lead to deadlocks, starvations, or other undesired 
effects\cite{multithread}.

To overcome this problem, Scala makes extensive use of the actor model, where 
basically only a single actor (thread) ever accesses the resource and 
the other actors query this actor by sending messages, and eventually waiting 
for responses from it. An actor assures to treat only a single message 
at a time, and to treat messages in order they arrive. If a message 
arrives while the actor is treating another message, this message will 
be queued, and treated afterwards. By using this model, access to a 
resource can be controled very easily.

Of course, Scala is not the only language using the actor model. To 
give an example between others, Erlang uses this communication system 
extensively. Furthermore, the library used in this program, Akka, is
also available for Java, however the impossibility of writing DSL like 
code in Java due to the inexitance of operator overloading makes the 
general syntax less consise and harder to read.
%
\subsection{Inter-process communication}
This system uses a lot of Inter-process communication(IPC) to communicate between the backend and frontend
applications on the server machine. How easy this communication is made was 
also an important criteria in the choice of the language and library.

Based on the above model, inter-process is made very easy, 
using the same model over a TCP connection. The Akka library allows a 
great abstraction for the programmer, and while IPC would 
generally require to manually serialize data, and make
 some other verification  can be done in a completely transparent fashion.
