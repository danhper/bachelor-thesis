\chapter{Web browser application}
%
The web browser application is the interface through which the user connects to the system 
and then controls it. It receives messages from the server frontend application and renders them 
on the browser screen. It also handles the browser's events and transfer them to frontend
application.
%
\section{Rendering}
%
As mentioned above, the first role of the application is to render the requests received from 
the frontend application, and therefore indirectly from the X client running for the user's 
X session. Listing~\ref{list:sample-request} shows a sample request that is received 
through the websocket channel and rendered by the application.
\begin{lstlisting}[basicstyle=\footnotesize,caption=Sample request,language=javascript,label=list:sample-request]
{
  "type":"request",
  "content": {
    "clientId": 1,
    "opCode": 70,
    "action": "PolyFillRectangleRequest",
    "request": {
      "drawable": 4194305,
      "context": 4194304,
      "rectangles": [{"x":20,"y":20,"width":200,"height":200}]
    }
  }
}
\end{lstlisting}
The object contained in the \lstinline{request} key is almost exactly the request 
sent by the X client to the backend application, translated in JSON, except from the 
op-code being stored in the wrapping object to make the request easier to handle.
The rest of the message is extra meta information to help the browser handling 
requests efficiently. 