\section{Event handling}
%
The next of this application is event handling, however, the implementation of this 
functionality not being done yet, we will here only give a general overview of the 
approach that will be taken to implement it.

The event that are to be handled by the X server are decided by the X client, 
and are different for each window.

The windows are created as KineticJS groups. The framework allows each node to have 
its own event logic, and groups being node, it is possible to assign all the 
events that JavaScript allow to a particular window. Therefore, when a window 
is set to react to some events, an event handler aware of the window resource id 
could be added, sending the event to the frontend application when triggered.

However, for events such as mouse motions, which trigger with a very high frequency, 
the events could be queued and sent when no event is triggered during an arbitrary
period or when more than an arbitrary number of events have already been. This could 
help saving bandwidth and improving the general performances of the application.
