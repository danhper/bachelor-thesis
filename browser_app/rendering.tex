\section{Rendering}
%
As mentioned above, the first role of the application is to render the requests received from 
the frontend application, and therefore indirectly from the X client running for the user's 
X session. Listing~\ref{list:sample-request} shows a sample request that is received 
through the websocket channel and rendered by the application.
\begin{lstlisting}[basicstyle=\footnotesize,caption=Sample request,language=javascript,label=list:sample-request]
{
  "type":"request",
  "content": {
    "clientId": 1,
    "opCode": 70,
    "action": "PolyFillRectangleRequest",
    "request": {
      "drawable": 4194305,
      "context": 4194304,
      "rectangles": [{"x":20,"y":20,"width":200,"height":200}]
    }
  }
}
\end{lstlisting}
The object contained in the \lstinline{request} key is almost exactly the request 
sent by the X client to the backend application, translated in JSON, except from the 
op-code being stored in the wrapping object to make the request easier to handle.
The rest of the message is extra meta information to help the browser handling 
requests efficiently. 

The rendering implementation is based on KineticJS, and each window is treated as 
a different layer so that refreshing the display can be done without having to 
draw again the whole screen. This rendering strategy improves a lot performance as the
area to redraw for each request is only the given part of the window and the rest of 
the display can stay intact.
