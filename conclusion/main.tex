\chapter{Conclusion}
Though a lot of work still needs to be done, we have managed to 
build a system to use a Linux/UNIX machine from a web browser 
using the X protocol. 
We managed to model and partly implement a system which could be 
self-sufficient for any graphical access to a machine running only 
X clients as graphical applications, which could remove a lot of 
dependencies and complexity in some cases.

From this experience, we could also show that new web technologies 
offer a lot of new possibilities, and we will try to continuing to prove it 
while continuing the project.
%
\section{Discussion and future work}
There are a lot of points that need further confirmation, and we will
especially need to check performance in a real use-case, to see 
if the system could actually be used responding to real-world requirements.
However, as discussed in section~\ref{sec:performances}, we are expecting 
the performances to be enough for a daily life usage, and more than enough 
for educational purpose such as programming.

Some unsolved problems still remain, such as truetype fonts handling, 
or large data transfer. Truetype fonts could be rendered as drawings using 
Canvas
\footnote{The \href{http://typeface.neocracy.org/}{Typeface} project seems to be able to render truetype fonts using Canvas.}
. Large data transfer could be done opening another websocket channel 
to avoid blocking the main communication channel especially when the 
connection is slow.

This project will continue to be developed, and we will try to reach 
a real-world usable implementation in the next six months.
