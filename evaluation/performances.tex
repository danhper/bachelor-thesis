\section{Performances}
%
\subsection{Used technologies}
From a hardware level, the system performance should be good enough. We will 
here think the data transfer performances as well as the rendering performances.
%
\subsubsection{Data transfer and bandwidth}
%
Bandwidth has been increasing impressively during the last couple of years. 
About 10 years ago, $512$kbits/s ADSL was quite common, and the downstream and 
upstream rate was very limited. 

However, with the improvement of the communication protocols and the 
spread of fiber-optic communication, these rate has become higher and higher.
Using fiber-optic communication, a server 
%
\subsubsection{Web browser rendering}
%
Until HTML4, rendering something else than text in a web browser required the 
use of some other technologies, such as flash or yet Java applets. 

From HTML5, Canvas has appeared in web browser and bitmap rendering can now 
be done natively. Furthermore, JavaScript implementations
are becoming faster and faster, and recent JavaScript implementations 
are way faster than flash or Java applets\cite{js-vs-as}.

In this system, the main purpose of the X server working is to be able to use
a Linux distribution in a web browser, mainly for a normal desktop use and 
not to play 3D games in it. This should hardly require more than 10 frames per second, 
except maybe for videos. With the current performance of JavaScript and 
recent processors, such a frame rate should be easy enough to reach, and we 
therefore think that rendering performances should not be too much of a bottleneck 
for a main stream usage of the system.
%
\subsection{System implementation}
%
The system not yet at a very advanced implementation state, but there is however a point that 
needs to be fixed in order to improve the general performances.

In the actual implementation of the system, the server frontend application takes care of 
handling the client login, but also acts as a bridge between the backend application and the 
web browser. That means that all requests and replies pass through this application.

This is bad, and will therefore be fixed in the future releases, for two reasons.
\begin{itemize}
\item The requests and replies need to be transferred two times, and therefore have to 
  be sent over three different channels to go from an X client to the web browser as shown 
  in figure~\ref{fig:project-overview}.
\item If several users connect at the same time, the burden of the frontend 
  application could become too big. This would slow down all the clients.
\end{itemize}
This could be greatly improved by having the web browser communicating 
directly with the backend application after the login phase.

For this improvement, we would simply need to have the frontend application 
sending the authentication token to the backend application after starting it.
The web browser application would then been sent the hostname and eventually port 
to connect to. When connecting to the backend application which would check the 
received token and then finish setting up the connection.
