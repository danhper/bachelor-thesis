\section{Motivations}
\subsection{Easy access to Linux}
From its first releases, and during several years, Linux has been quite 
complicated and not very intuitive to use (eg. no graphic installer) for beginners.
In the last few years, Linux has become much simpler to install, and use in general, 
with distributions such as Ubuntu or Fedora integrating a user friendly installer 
and even tools such as Wubi\footnote{http://www.ubuntu.com/download/desktop/windows-installer} 
to make the installation possible from Windows.

However, even if the installation process has become much simpler, the fact is that 
a lot of beginners are still having a lot of problems to get a usable Linux environment. 
While teaching C programming language as a teacher assistant, I wrote 
on my personal blog all the steps to get a usable Linux environment, using the 
possibly simplest setup: giving a disk image file to open with VirtualBox. 
Despite this, a large number of students still could not get Linux to work properly.

One of the greatest motivation for this project was therefore to create a system to 
make Linux available to anyone, even with no computer knowledge at all.
%
\subsection{Access from mobile devices}
Another interesting possibility for this project is a mobile access from any modern 
phone or tablet to a remote machine. By using this system, one could check anything,
for example the status of a running task,  on a given machine without having 
to create or use a dedicated API for it. This could be useful when the creation of 
a dedicated tool is not worth.
%
\subsection{Evaluation of new web technologies }
Web technologies have been evolving during the last few years, and a lot 
of new tools and technologies have been released. With the introduction of
websockets, a full duplex communication between a browser and a web server 
have become possible. 
The last motivation for this project was to try and evaluate these new web 
technologies, in particular websockets, to see how well it can be integrated 
in a resource demanding project.

