\section{Example}
To end up with the presentation of the X protocol, we will here give a
short example of a communication between an X server and an X client. 
In this example, we will explain what happens at the protocol level 
when the following code, using Xlib\footnote{Xlib is a library to implement client
application using the X protocol. It takes care of all the low-level details of the 
protocol, such as byte-swapping, or choosing ids for allocated resources.}, is ran.
%
\lstinputlisting[basicstyle=\footnotesize, 
     caption=Simple example of an X application,stepnumber=2]{about_x/x_test.c}
%
\subsection{Connection setup}
On line $14$, the client tries to open the screen $0$ of display number $2$
for the X server running on \lstinline{localhost}. In the X window system, for TCP connections,
like in this case, the display number $0$ should be listening on the port $6000$, and 
other display should increase the port number by $1$. Therefeore, the display number $2$ is 
supposed to be listening port $6002$. If no connection can be established to this host and port, 
the program fails. If the connection is established, the client send connection information to 
the server. The client first specifies the byte-order (LSBF or MSBF), then the major and 
minor version number of the used X protocol, and finally if needed, authorization protocol 
name and data.
%
If the connection is rejected, the server sends back a message indicating this, as well 
as the reason for rejecting the connection.
%
If the connection is accepted, the server sends back a message with the necessary server and 
screen information.
%
\subsection{Requests}
The \lstinline{XOpenDisplay} function actually offers some abstraction for the end user and
does not only initialize the connection but also query for extensions, and setup a basic 
graphic context. 

The first request to be sent is a \lstinline{QueryExtension} request, querying for the 
\lstinline{BigRequest}\footnote{The \lstinline{BigRequest} extension is 
  an extension to allow requests of more than the 262140 bytes.} extension. The client then blocks 
until receiving a reply from the server. This is however not decided by the protocol itself, 
and depending on the request and reply involved, the client may choose to send the request 
asynchronously.

The next request to be sent is the request to create a default graphic context. 
The created graphic context is the one that will be used when the client uses the 
\lstinline{DefaultGC (Display*, int)} function.

The client then creates a window, using the information received during the connection 
initialization, such as the root window and the black and white pixels information. This 
generate a \lstinline{CreateWindow} request which does not need a reply, the client therefore 
sends the next request straight away. 

\lstinline{XSelectInput} is used to select the kind of events the client is listening. 
This is important to avoid receiving events that will never be hanlded. However, this request does 
not exist in the X protocol and \lstinline{ChangeWindowAttributes} request with the 
\lstinline{event-mask} set with the proper value is sent instead.

Finally, a \lstinline{MapWindow} request is sent to map the created window on the screen.
%
\subsection{Events}
The program works around a simple event loop, in which only two kinds of events are 
handled: the \lstinline{Expose} and \lstinline{KeyPress} events. 
The \lstinline{XNextEvent} is a function that blocks the execution of the program 
until an event is received from the server, and set the event when received.

The first received event should be the \lstinline{Expose} event, which is generated when
a window is shown on the screen, so that should be generated if the \lstinline{MapWindow} request 
was handled properly.

The other handled event is the \lstinline{Keypress} event, which is generated whenever 
a key is pressed on the keyboard the server is listening. In this example, 
the \lstinline{Keypress} event is not really handled, in the sense that neither the keycode nor 
the keysims are checked and the program simply breaks from the event loop, but the event does 
contain all the necessary information in order to properly handle it, and those information 
are usually indeed checked in a real world application.
%
\subsection{Closing the display}
As the client creates a number of resources while running, here a graphic context and a window, 
those needs to be destroyed when the client program ends. The \lstinline{XCloseDisplay} is 
a high-level function that takes care to send the appropriate \lstinline{DestroyWindow} and 
\lstinline{FreeGC} requests to the server in order for the resources to be freed properly.