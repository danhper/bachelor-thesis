\section{Overview}
%
First of all, we are going to briefly explain why and how can this model be useful, 
and then explain how it works.
%
\subsection{Motivations}
%
The first motivation for the actor model is too make multi-threading simpler and safer.
To give an overview of the problem, in listing~\ref{list:pthread-example}
we are going to take a short example written in C++ using the pthread library for threading.
%
\begin{lstlisting}[basicstyle=\footnotesize,caption=Threading example in C++,language=C++,label=list:pthread-example]
bool resource_manager::get_resource(int id, resource& r)
{
  pthread_mutex_lock(&this->_resource_lock);
  if(!this->has_resource(id)) {
    return false;
  }
  r = this->_resource[id];
  pthread_mutex_unlock(&this->_resource_lock);
  return true;
}

// in some other method
if(!manager->get_resource(wanted_id, wanted_resource)) {
  // if wanted id not available get fallback instead
  manager->get_resource(fallback_id, wanted_resource);
}
\end{lstlisting}
In the above example, the application will deadlock as soon as the \lstinline{wanted_id}
is not available in the resource manager, as the lock is not released in that case.

Of course, in a simple example like above, any programmer having a little experience 
with threads would notice immediately that the lock is not released and it would 
probably would not take too long to debug anyway as the deadlock occurs each time the 
wanted id does not exist. 
However, everything is not always so simple, and as timing plays an important 
role when working with threads, deadlocks can be very difficult to debug.

An important motivation for using the actor model is to give an extra layer of 
abstraction to avoid this kind of problem.
%
\subsection{Workflow with actors}
When working with actors, the basic rule is that actor communicate 
with each others only using messages, and the actors' data should (almost) never be 
used and even less modified directly.

The main reason for this rule is that messages are treated sequentially by the 
actor, which is enough to avoid having to lock the data each time we try to use it, 
with the only condition being to respect the above rule.

We will give a short example of the above example in using the akka library in Scala.
%
\begin{lstlisting}[basicstyle=\footnotesize,caption=Threading using actors,label=list:actor-example,language=scala]
class ResourceManager extends Actor {
  private var resources: Map[Int, Resource] = Map.empty
  def receive = {
    case ResourceRequest(resId) => resources.get(resId) match { 
      case None => sender ! NotFound
      case Some(resource) => sender ! ResourceReply(resource)
    }
    ...
  }
  ...
}
\end{lstlisting}
Each time an actor receives a message, the \lstinline{receive} method is executed, 
the common pattern is therefore to match the different messages that could possibly 
received and to handle it.

In the above example, when an other actor requests a resource, the \lstinline{ResourceManager} 
actor checks if the resource exists, if it does it returns the resource to the sender 
wrapped in a \lstinline{ResourceReply} object to help the receiver handling the 
response more easily, otherwise it returns a \lstinline{NotFound} object.

The reason why no lock is needed here is, as explained above, that messages are treated 
sequentially and that actor resources are only accessed through messages, therefore 
two threads will never be accessing the same resource at the same time, which 
simplifies a lot synchronization.
