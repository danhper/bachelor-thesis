\section{Scala integration}
The actor pattern integrates very well with Scala for different reasons that 
we will briefly discuss about.
\subsection{Pattern matching}
Scala allows very extensive pattern matching. As opposed to \lstinline{switch} 
statements in a lot of imperative languages, in which only a few predefined types 
can be match, Scala allows, through \lstinline{case class} and \lstinline{case object}
to match much anything. A \lstinline{case class} is an immutable class that have the 
necessary features to be matched. A \lstinline{case object} is its singleton version.

This makes message handling when using actor very simple and efficient. 
To give a short example of it, let think of a chat client which can receive messages 
when a user connects, disconnects, or when a message is received. 
The case classes to use could be defined as in~\ref{list:chat-example-def} and the 
method handling could be implemented as in in~\ref{list:chat-example-impl}.
%
\begin{lstlisting}[basicstyle=\footnotesize,caption=Chat example definitions,label=list:chat-example-def,language=scala]
case class UserConnection(username: String)
case class UserDisconnection(username: String)
case class Message(username: String, message: String)
\end{lstlisting}
%
\begin{lstlisting}[basicstyle=\footnotesize,caption=Chat example implementation,label=list:chat-example-impl,language=scala]
class ChatActor extends Actor {
  private var users: Set[String] = Map.empty
  def receive = {
    case UserConnection(username) => {
      users += username
      chatView.putInfo(username + " has connected.")
    }
    case UserDisconnection(username) => {
      users -= username
      chatView.putInfo(username + " has disconnected.")
    }
    case Message(username: String, message: String) => {
      chatView.putMessage(username, message)
    }
  }
}
\end{lstlisting}
%
This example if of course very minimalist but reflects quite well the simplicity of the 
model. Furthermore, the akka library integrates remote actors, which can be used to have 
the class handling a TCP connection. If the server sending the messages is using the same 
library and the same class definitions, the messages can be passed through TCP only by 
changing a configuration file.
%
\subsection{Domain Specific Language}
This part is a small details but plays a role in the general source code readability.
