\section{Caching}
%
As said in the previous section, an important responsibility of the backend application is to
cache data received by both the X clients and the frontend application, and to use this cache 
efficiently to try to reduce the most possible the requests and replies transfers to the 
frontend application.

The caching system has not yet been implemented in the system when writing this paper, but the 
strategy to use is straightforward and should not be too difficult to implement.

The first step is to check if the requests content needs to be cached or not. 
As the requests are implemented as Scala case classes, we could easily add a trait 
to the requests that need to be cached. A possible example of this implementation is shown 
in listing \ref{list:caching}.

\begin{lstlisting}[basicstyle=\footnotesize,caption=Possible implementation of requests needing cached,language=scala,label=list:caching]
// define trait for caching
trait NeedsCaching {
  def cacheRequest: Unit
}

// mixin trait to class needing to be cached
case class CreateWindowRequest (
...
) extends Request 
  with NeedsTransfer // needs to be transfered to the frontend
  with NeedsCaching { // needs to be cached 
  ...
  def cacheRequest {
    // do cache logic
  }
}
\end{lstlisting}

